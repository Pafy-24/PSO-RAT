\documentclass[conference]{IEEEtran}
\IEEEoverridecommandlockouts
% The preceding line is only needed to identify funding in the first footnote. If that is unneeded, please comment it out.
\usepackage{cite}
\usepackage{amsmath,amssymb,amsfonts}
\usepackage{algorithmic}
\usepackage{graphicx}
\usepackage{textcomp}
\usepackage{xcolor}
\def\BibTeX{{\rm B\kern-.05em{\sc i\kern-.025em b}\kern-.08em
    T\kern-.1667em\lower.7ex\hbox{E}\kern-.125emX}}
\begin{document}

\title{PSO-RAT: Implementarea unui Remote Administration Tool pentru Proiectarea Sistemelor de Operare}

\author{\IEEEauthorblockN{Romaș Ștefan-Sebastian}
\IEEEauthorblockA{\textit{Facultatea de Calculatoare și sisteme informatice de securitate și apărare națională} \\
\textit{Academia Tehnică Militară ,,Ferdinand I"}\\
București, România \\
stefan-sebastian.romas@stud.mta.ro}
}

\maketitle

\begin{abstract}
Acest document prezintă implementarea unui Remote Administration Tool (RAT) educațional dezvoltat în cadrul cursului de Proiectarea Sistemelor de Operare. Proiectul demonstrează aplicarea practică a conceptelor fundamentale de sisteme de operare: socket-uri TCP/UDP pentru comunicare în rețea, thread-uri pentru execuție concurentă, mecanisme de sincronizare (mutex) și gestionarea proceselor prin fork/exec. Arhitectura client-server implementată utilizează comunicare JSON și permite controlul remote al stațiilor client prin comenzi administrative. Implementarea în C++17 folosește RAII și smart pointers pentru gestionarea automată a memoriei.
\end{abstract}

\begin{IEEEkeywords}
sisteme de operare, socket-uri, thread-uri, mutex, pipe-uri, client-server, RAT, C++17, POSIX
\end{IEEEkeywords}

\section{Introducere}

Remote Administration Tools (RAT) sunt aplicații software care permit controlul și administrarea la distanță a sistemelor de calcul. Deși astfel de instrumente pot fi utilizate în contexte legitime (administrare IT, suport tehnic), acest proiect își propune să demonstreze conceptele fundamentale ale sistemelor de operare într-un context educațional controlat.

Proiectul PSO-RAT a fost dezvoltat în cadrul cursului de Proiectarea Sistemelor de Operare și implementează o arhitectură client-server care demonstrează:

\begin{itemize}
\item Programarea socket-urilor TCP și UDP pentru comunicare în rețea
\item Arhitectură multi-threading cu sincronizare prin mutex
\item Utilizarea pipe-urilor pentru execuția comenzilor shell
\item Gestionarea proceselor (fork, exec, wait)
\item Management automat al memoriei prin RAII și smart pointers
\item Comunicare structurată prin protocol JSON
\end{itemize}

Obiectivul principal este aplicarea practică a conceptelor teoretice învățate în cadrul cursului PSO, cu accent pe corectitudinea implementării și respectarea principiilor de programare sistem.

\section{Arhitectura Sistemului}

\subsection{Prezentare Generală}

Sistemul este organizat într-o arhitectură client-server clasică, cu un server central care gestionează conexiuni multiple de la clienți și oferă o interfață administrativă pentru controlul acestora. Comunicarea se realizează prin socket-uri TCP pentru transfer de date și UDP pentru mecanismul de keep-alive.

\subsection{Componente Server}

Serverul este structurat în mai multe module specializate:

\textbf{ServerManager} — Singleton care coordonează întregul server. Gestionează listener-ul TCP pentru acceptarea clienților noi și menține referințe către toate controller-ele active. Implementează pattern-ul RAII pentru cleanup automat al resurselor la închidere.

\textbf{ClientManagement} — Modul centralizat responsabil cu gestionarea thread-safe a clienților conectați. Folosește \texttt{std::mutex} pentru protejarea accesului concurrent la harta de clienți și implementează generarea automată de nume unice pentru fiecare client conectat.

\textbf{Controller-e Specializate}:
\begin{itemize}
\item \textit{ServerCommandController} — Interfață CLI administrativă care rulează într-un thread dedicat pentru citirea comenzilor din stdin
\item \textit{ServerFileController} — Gestionează transferul bidirecțional de fișiere cu encoding Base64
\item \textit{ServerPingController} — Implementează mecanismul de keep-alive prin UDP
\item \textit{ServerLogController} — Centralizează logging-ul thread-safe în \texttt{/tmp/rat\_server.log}
\end{itemize}

\subsection{Componente Client}

\textbf{ClientManager} — Orchestratorul principal care gestionează conexiunea la server și routează mesajele primite către controller-ele corespunzătoare.

\textbf{Controller-e Client}:
\begin{itemize}
\item \textit{ClientBashController} — Execută comenzi shell folosind \texttt{popen()} și returnează stdout/stderr
\item \textit{ClientFileController} — Implementează operațiile de upload/download
\item \textit{ClientScreenshotController} — Capturează screenshot-uri folosind utilitare sistem (scrot, import, gnome-screenshot)
\item \textit{ClientPingController} — Răspunde la request-urile de keep-alive
\item \textit{ClientKillController} — Gestionează deconectarea graciosa
\end{itemize}

\subsection{Utilități Comune}

Modulul \texttt{Utils} conține wrapper-e peste API-ul SFML Network:
\begin{itemize}
\item \texttt{TCPSocket} — Wrapper pentru socket-uri TCP cu API simplificat
\item \texttt{UDPSocket} — Wrapper pentru socket-uri UDP
\item \texttt{Base64} — Librărie header-only pentru encoding/decoding
\end{itemize}

\section{Concepte de Sisteme de Operare Implementate}

\subsection{Socket-uri TCP și UDP}

Comunicarea în rețea este realizată prin SFML Network, folosind două tipuri de socket-uri:

\textbf{Socket-uri TCP} — Folosite pentru comunicarea principală client-server. Serverul deschide un listener pe portul 5555 configurat în mod non-blocking pentru a accepta conexiuni noi fără a bloca execuția. Socket-urile client sunt configurate în mod blocking cu timeout de 5 secunde pentru a preveni deadlock-urile.

\textbf{Socket-uri UDP} — Implementează mecanismul de keep-alive (ping/pong) pentru detectarea clientilor deconectați. Alegearea UDP este justificată de natura stateless și overhead-ul redus pentru pachete mici periodice.

Implementarea wrapper-elor \texttt{TCPSocket} și \texttt{UDPSocket} demonstrează încapsularea API-ului de nivel jos și oferă interfețe simplificate pentru operațiile de \texttt{bind()}, \texttt{listen()}, \texttt{accept()}, \texttt{connect()}, \texttt{send()} și \texttt{receive()}.

\subsection{Thread-uri și Programare Concurentă}

Arhitectura multi-threading este esențială pentru funcționarea asincronă a serverului:

\begin{itemize}
\item \textbf{Thread principal} — Rulează loop-ul de acceptare clienți, blocând în \texttt{accept()} până la conectarea unui client nou
\item \textbf{Thread CLI} — \texttt{ServerCommandController} rulează un thread dedicat care citește comenzi din stdin, permițând interacțiune administrativă în timp ce serverul procesează clienți
\item \textbf{Thread-uri keep-alive} — Fiecare client poate avea thread-uri de ping pentru monitorizarea conexiunii
\end{itemize}

Crearea thread-urilor se face prin \texttt{std::thread}, demonstrând alternativa C++ la POSIX pthread. La distrugerea obiectelor, thread-urile sunt oprite prin flag-uri atomice și sincronizate cu \texttt{join()} pentru a evita resource leaks.

\subsection{Sincronizare prin Mutex}

Accesul concurrent la resurse partajate este protejat prin \texttt{std::mutex}:

\textbf{ClientManagement} folosește un mutex pentru protejarea hărții \texttt{clients\_} care stochează socket-urile active. Operațiile \texttt{addClient()}, \texttt{removeClient()} și \texttt{getClient()} sunt atomic executate sub lock.

\textbf{ServerManager} protejează queue-ul de log-uri \texttt{logs\_} cu un mutex dedicat \texttt{logMtx\_}, asigurând că scrierea concurentă de mesaje de la mai multe thread-uri nu produce corupție de date.

Pattern-ul folosit este \texttt{std::lock\_guard} pentru RAII — mutexul este automat eliberat la ieșirea din scope, chiar și în cazul excepțiilor.

\subsection{Pipe-uri și Execuție Comenzi}

\texttt{ClientBashController} demonstrează utilizarea pipe-urilor pentru execuția comenzilor shell și capturarea output-ului:

\begin{verbatim}
FILE* pipe = popen(cmd.c_str(), "r");
if (!pipe) return error;

char buffer[256];
while (fgets(buffer, sizeof(buffer), 
       pipe) != nullptr) {
    output += buffer;
}

int status = pclose(pipe);
int exitCode = WEXITSTATUS(status);
\end{verbatim}

Funcția \texttt{popen()} creează un proces copil, execută comanda într-un shell și returnează un file descriptor conectat la stdout-ul procesului. Citirea se face printr-un pipe implicit creat de sistem.

\subsection{Gestionarea Proceselor}

\texttt{ServerCommandController::spawnTerminal()} demonstrează utilizarea \texttt{fork()} și \texttt{exec()} pentru lansarea terminalelor externe:

\begin{verbatim}
pid_t pid = fork();
if (pid == 0) {  // proces copil
    setsid();     // detașare de terminal
    int devnull = open("/dev/null", 
                        O_WRONLY);
    dup2(devnull, STDOUT_FILENO);
    dup2(devnull, STDERR_FILENO);
    close(devnull);
    
    execlp("qterminal", "qterminal", 
           "-e", cmd.c_str(), NULL);
    _exit(127);   // fallback
}
// proces părinte continuă
\end{verbatim}

Procesul copil se detașează prin \texttt{setsid()}, redirecționează stdout/stderr către \texttt{/dev/null} și execută noul program prin \texttt{execlp()}. Procesul părinte returnează imediat, permițând execuția asincronă.

\subsection{Gestionarea Memoriei}

Proiectul demonstrează management modern al memoriei în C++17:

\textbf{Smart Pointers}:
\begin{itemize}
\item \texttt{std::unique\_ptr} — Pentru ownership exclusiv (socket-uri, controller-e). La distrugerea obiectului, resursele sunt automat eliberate.
\item \texttt{std::shared\_ptr} — Pentru \texttt{ServerManager} (singleton) și câteva resurse partajate între thread-uri.
\end{itemize}

\textbf{RAII Pattern} — Toate clasele implementează constructori/destructori care alocă/eliberează resurse automat, eliminând necesitatea \texttt{delete} manual și prevenind memory leaks.

Exemplu de RAII în \texttt{ServerManager}:
\begin{verbatim}
ServerManager::~ServerManager() {
    stop();  // oprește thread-uri
    std::lock_guard<std::mutex> 
        lock(mtx_);
    cleanupResources();  // închide 
                         // socket-uri
}
\end{verbatim}

\section{Protocol de Comunicare}

\subsection{Formatul Mesajelor}

Comunicarea client-server folosește JSON (nlohmann::json library) pentru structurarea mesajelor. Alegerea JSON este motivată de:
\begin{itemize}
\item Lizibilitate — mesajele pot fi debug-ate ușor
\item Flexibilitate — schema poate fi extinsă fără breaking changes
\item Suport nativ C++ — librăria nlohmann oferă interfață idiomatică
\end{itemize}

Structura generală a unui mesaj:
\begin{verbatim}
{
  "controller": "bash|file|screenshot|...",
  "action": "execute|download|upload|...",
  ...parametri specifici...
}
\end{verbatim}

\subsection{Pattern Request-Response Sincron}

Serverul implementează comunicare sincronă prin două metode:

\texttt{sendRequest(clientName, json)} — Trimite cererea către client și returnează imediat status (success/failure pentru trimitere)

\texttt{receiveResponse(clientName, json\&, timeout)} — Blochează până primește răspuns de la client sau expiră timeout-ul (5-30 secunde în funcție de operație)

Acest pattern simplifică logica și elimină necesitatea callback-urilor asincrone, potrivit pentru scopul educațional al proiectului.

\subsection{Înregistrarea Clientului}

La conectare, serverul trimite un request de tip \texttt{"register"}:
\begin{verbatim}
Server -> Client: 
  {"type": "register"}
  
Client -> Server:
  {
    "hostname": "arch-desktop",
    "os": "Linux",
    "kernel": "6.1.0-arch1-1"
  }
\end{verbatim}

Serverul generează un nume unic (ex: \texttt{arch\_0}) și adaugă clientul în \texttt{ClientManagement}.

\section{Funcționalități Demonstrate}

\subsection{Execuție Comenzi Remote}

Comanda \texttt{bash} demonstrează utilizarea pipe-urilor:

\begin{verbatim}
Admin: bash arch_0 ls -la /tmp

Server -> Client:
  {
    "controller": "bash",
    "cmd": "ls -la /tmp"
  }

Client execută prin popen()

Client -> Server:
  {
    "output": "total 48\ndrwxr-xr-x...",
    "exitcode": 0
  }

Admin: vede output-ul
\end{verbatim}

\subsection{Transfer Fișiere}

Operațiile \texttt{download} și \texttt{upload} folosesc Base64 encoding pentru transfer binar în JSON:

\textbf{Download}:
\begin{verbatim}
Admin: download arch_0 /etc/hostname ./h.txt

Server -> Client:
  {
    "controller": "file",
    "action": "download",
    "path": "/etc/hostname"
  }

Client citește fișier + Base64::encode()

Client -> Server:
  {
    "success": true,
    "data": "YXJjaC1kZXNrdG9wCg==",
    "size": 13,
    "filename": "hostname"
  }

Server: Base64::decode() + write to ./h.txt
\end{verbatim}

\textbf{Upload} funcționează invers — serverul citește fișierul local, encode-ază și trimite către client care decode-ază și scrie.

Limitare: Fișiere max 50MB pentru a evita overflow-ul memoriei.

\subsection{Screenshot Remote}

Funcționalitatea de screenshot demonstrează integrarea cu utilitare sistem:

\begin{verbatim}
Admin: screenshot arch_0

Server -> Client:
  {
    "controller": "screenshot",
    "action": "take"
  }

Client:
  1. Detectează utilitar disponibil
     (scrot/import/gnome-screenshot)
  2. system("scrot /tmp/rat_screenshot_
     <timestamp>.png")
  3. Citește PNG ca binar
  4. Base64::encode()

Client -> Server:
  {
    "success": true,
    "data": "iVBORw0KGgoAAAANS...",
    "size": 145230
  }

Server:
  1. Base64::decode()
  2. Write to /tmp/ss_arch_0_<time>.png
  3. system("xdg-open <file> &")
\end{verbatim}

\subsection{Keep-Alive Mechanism}

\texttt{ServerPingController} menține conexiunea prin UDP ping/pong periodic (interval 30s). Dacă un client nu răspunde în 3 încercări consecutive, este marcat ca deconectat și eliminat din \texttt{ClientManagement}.

Implementarea UDP pentru ping reduce overhead-ul comparativ cu TCP și nu afectează transmisia de date principală.

\begin{table}[htbp]
\caption{Comenzi Administrative Implementate}
\begin{center}
\begin{tabular}{|p{2.5cm}|p{5cm}|}
\hline
\textbf{Comandă} & \textbf{Descriere} \\
\hline
\texttt{list} & Listează clienții conectați cu IP-uri \\
\hline
\texttt{choose <client>} & Selectează client implicit pentru comenzi ulterioare \\
\hline
\texttt{bash <client> <cmd>} & Execută comandă shell și returnează output \\
\hline
\texttt{download <client> <remote> <local>} & Descarcă fișier de la client \\
\hline
\texttt{upload <client> <local> <remote>} & Încarcă fișier pe client \\
\hline
\texttt{screenshot <client>} & Capturează și afișează screenshot \\
\hline
\texttt{kill <client>} & Deconectează client specific \\
\hline
\texttt{killall} & Deconectează toți clienții \\
\hline
\texttt{showlogs} & Deschide terminal cu tail -F pe log \\
\hline
\texttt{quit} & Oprește serverul \\
\hline
\end{tabular}
\label{tab1}
\end{center}
\end{table}

\section{Testare și Rezultate}

\subsection{Mediu de Testare}

Sistemul a fost testat pe:
\begin{itemize}
\item \textbf{OS}: Arch Linux 6.1.0, Ubuntu 22.04 LTS
\item \textbf{Compilator}: GCC 13.2.1, C++17
\item \textbf{Librării}: SFML 2.6.0, nlohmann-json 3.11.2
\end{itemize}

\subsection{Scenarii de Testare}

\textbf{Test 1: Conectare multiplă}
\begin{itemize}
\item Pornit 5 clienți simultan
\item Verificat thread-safety în ClientManagement
\item Rezultat: Toți clienții înregistrați corect, fără race conditions
\end{itemize}

\textbf{Test 2: Execuție comenzi concurente}
\begin{itemize}
\item Trimis comenzi bash către 3 clienți în paralel
\item Verificat că output-urile nu se amestecă
\item Rezultat: Fiecare răspuns asociat corect cu clientul său
\end{itemize}

\textbf{Test 3: Transfer fișiere mari}
\begin{itemize}
\item Upload fișier 45MB (sub limita de 50MB)
\item Verificat integritate prin checksum MD5
\item Rezultat: Transfer reușit, date integre
\end{itemize}

\textbf{Test 4: Deconectare bruscă}
\begin{itemize}
\item Închis forțat client (kill -9)
\item Verificat cleanup resurse pe server
\item Rezultat: Socket închis corect, memory leak-uri absent (verificat cu valgrind)
\end{itemize}

\textbf{Test 5: Screenshot multi-display}
\begin{itemize}
\item Testat pe sistem cu 2 monitoare
\item Rezultat: Capturează corect ecranul principal
\end{itemize}

\subsection{Metrici de Performanță}

\begin{itemize}
\item \textbf{Latență medie}: 25-50ms pentru comenzi bash simple (ping localhost)
\item \textbf{Throughput transfer}: \textasciitilde8-10 MB/s pentru fișiere (limitat de overhead Base64 \textasciitilde33\%)
\item \textbf{Memorie server}: \textasciitilde15MB RAM cu 5 clienți conectați
\item \textbf{CPU idle}: <1\% când nu procesează cereri active
\end{itemize}

\section{Limitări și Îmbunătățiri Viitoare}

\subsection{Limitări Actuale}

\textbf{Securitate}: Comunicarea este în plain-text fără autentificare sau criptare. Potrivit doar pentru medii educaționale controlate în rețele locale de încredere.

\textbf{Scalabilitate}: Arhitectura sincronă nu suportă eficient sute de clienți simultan. Fiecare request blochează până la răspuns.

\textbf{Error Recovery}: Dacă un client se blochează, serverul așteaptă timeout-ul complet înainte să continue.

\textbf{Protocol Rigid}: Schema JSON nu are versioning, upgrade-uri viitoare ar putea sparge compatibilitatea.

\subsection{Îmbunătățiri Propuse}

\begin{itemize}
\item \textbf{TLS/SSL}: Adăugare criptare prin OpenSSL pentru trafic securizat
\item \textbf{Autentificare}: Token-based authentication cu expire time
\item \textbf{Asincron I/O}: Refactoring către epoll/io\_uring pentru scalabilitate
\item \textbf{Thread Pool}: Înlocuirea thread-urilor dedicate cu un pool de worker threads
\item \textbf{Logging Structured}: JSON logs pentru parsing automat și monitoring
\item \textbf{Protocol Buffers}: Înlocuirea JSON cu protobuf pentru overhead redus
\end{itemize}

\section{Concluzii}

Proiectul PSO-RAT demonstrează aplicarea practică a conceptelor fundamentale de sisteme de operare într-un context real și funcțional. Implementarea în C++17 respectă principiile moderne de programare sistem: RAII pentru management resurse, smart pointers pentru eliminarea memory leaks, și utilizarea API-urilor POSIX pentru operații de nivel jos.

Conceptele PSO acoperite includ:
\begin{itemize}
\item Socket programming (TCP/UDP) pentru comunicare în rețea
\item Multi-threading cu sincronizare prin mutex
\item Pipe-uri pentru comunicare inter-proces
\item Gestionarea proceselor (fork, exec, wait, signals)
\item Memory management prin smart pointers și RAII
\end{itemize}

Deși sistemul are limitări de securitate și scalabilitate (inerente unui proiect educațional), arhitectura modulară permite extindere și îmbunătățire viitoare. Codul sursă este disponibil pe GitHub pentru studiu și experimentare în medii controlate.

Proiectul a oferit o înțelegere profundă a modului în care conceptele teoretice de sisteme de operare se traduc în implementări practice și provocările întâlnite în dezvoltarea de software sistem real.

\section*{Acknowledgment}

Acest proiect a fost realizat în cadrul cursului Proiectarea Sistemelor de Operare, semestrul 2, anul universitar 2025-2026, la Facultatea de Automatică și Calculatoare, Universitatea Politehnica București.

Mulțumiri cadrelor didactice pentru ghidarea conceptelor de sisteme de operare și comunității open-source pentru librăriile SFML și nlohmann::json care au facilitat implementarea.

\begin{thebibliography}{00}
\bibitem{tanenbaum2014modern}
A.~S.~Tanenbaum and H.~Bos, 
``Modern Operating Systems,'' 
4th ed., Pearson, 2014.

\bibitem{silberschatz2018operating}
A.~Silberschatz, P.~B.~Galvin, and G.~Gagne,
``Operating System Concepts,''
10th ed., Wiley, 2018.

\bibitem{stevens2013advanced}
W.~R.~Stevens and S.~A.~Rago,
``Advanced Programming in the UNIX Environment,''
3rd ed., Addison-Wesley, 2013.

\bibitem{kerrisk2010linux}
M.~Kerrisk,
``The Linux Programming Interface: A Linux and UNIX System Programming Handbook,''
No Starch Press, 2010.

\bibitem{sfmlnetwork}
SFML Development Team,
``SFML Network Module Documentation,''
[Online]. Available: https://www.sfml-dev.org/documentation/2.6.0/group\_\_network.php

\bibitem{nlohmannjson}
N.~Lohmann,
``JSON for Modern C++,''
[Online]. Available: https://json.nlohmann.me/

\bibitem{posixthreads}
IEEE and The Open Group,
``POSIX Threads Programming,''
[Online]. Available: https://pubs.opengroup.org/onlinepubs/9699919799/

\bibitem{cppref}
C++ Reference,
``C++ Standard Library,''
[Online]. Available: https://en.cppreference.com/

\end{thebibliography}

\end{document}
