\documentclass[conference]{IEEEtran}
\IEEEoverridecommandlockouts
\usepackage{cite}
\usepackage{amsmath,amssymb,amsfonts}
\usepackage{algorithmic}
\usepackage{graphicx}
\usepackage{textcomp}
\usepackage{xcolor}
\usepackage[romanian]{babel}
\usepackage[utf8]{inputenc}
\def\BibTeX{{\rm B\kern-.05em{\sc i\kern-.025em b}\kern-.08em
    T\kern-.1667em\lower.7ex\hbox{E}\kern-.125emX}}
\begin{document}

\title{PSO-RAT: Implementarea unui Remote Administration Tool\\pentru Proiectarea Sistemelor de Operare}

\author{\IEEEauthorblockN{Romaș Ștefan-Sebastian}
\IEEEauthorblockA{\textit{Facultatea de Calculatoare și Sisteme Informatice de Securitate și Apărare Națională} \\
\textit{Academia Tehnică Militară „Ferdinand I"}\\
București, România \\
stefan-sebastian.romas@stud.mta.ro}
}

\maketitle

\begin{abstract}
Acest document prezintă implementarea unui Remote Administration Tool (RAT) educațional dezvoltat în cadrul cursului de Proiectarea Sistemelor de Operare la Academia Tehnică Militară. Proiectul demonstrează aplicarea practică a conceptelor fundamentale de sisteme de operare: socket-uri TCP/UDP pentru comunicare în rețea, thread-uri pentru execuție concurentă, mecanisme de sincronizare prin mutex, gestionarea memoriei și proceselor prin fork/exec și utilizarea pipe-urilor pentru comunicare inter-proces. Arhitectura client-server implementată utilizează comunicare JSON și permite controlul remote al stațiilor client prin comenzi administrative. Implementarea în C++17 folosește smart pointers pentru gestionarea automată și sigură a memoriei.
\end{abstract}

\begin{IEEEkeywords}
sisteme de operare, socket-uri, thread-uri, mutex, pipe-uri, gestionarea memoriei, procese client-server, RAT, C++17, POSIX
\end{IEEEkeywords}

\section{Introducere}

Remote Administration Tools (RAT) sunt aplicații software care permit controlul și administrarea la distanță a sistemelor de calcul. Deși astfel de instrumente pot fi utilizate în contexte legitime (administrare IT, suport tehnic), acest proiect își propune să demonstreze conceptele fundamentale ale sistemelor de operare într-un context educațional strict controlat.

Proiectul PSO-RAT a fost dezvoltat în cadrul cursului de Proiectarea Sistemelor de Operare la Academia Tehnică Militară și implementează o arhitectură client-server complexă care demonstrează:

\begin{itemize}
\item \textbf{Gestionarea memoriei} — Smart pointers (unique\_ptr, shared\_ptr) pentru management automat al resurselor
\item \textbf{Pipe-uri} — Utilizarea popen() pentru execuție comenzi shell și capturare output
\item \textbf{Procese} — fork(), exec(), wait() pentru crearea și gestionarea proceselor copil
\item \textbf{Thread-uri} — Arhitectură multi-threading pentru execuție concurentă și non-blocking I/O
\item \textbf{Mutex} — Sincronizare thread-safe prin std::mutex și std::lock\_guard
\item \textbf{Socket-uri TCP/UDP} — Comunicare client-server prin SFML Network wrappers
\end{itemize}

Obiectivul principal este aplicarea practică a conceptelor teoretice învățate în cadrul cursului PSO, cu accent pe corectitudinea implementării și respectarea principiilor de programare sistem moderne.

\section{Arhitectura Sistemului}

\subsection{Prezentare Generală}

Sistemul este organizat într-o arhitectură client-server clasică, cu un server central care gestionează conexiuni multiple de la clienți și oferă o interfață administrativă pentru controlul acestora. Comunicarea se realizează prin socket-uri TCP pentru transfer de date și UDP pentru mecanismul de keep-alive.

\subsection{Componente Server}

Serverul este structurat în mai multe module specializate:

\textbf{ServerManager} — Singleton care coordonează întregul server. Gestionează listener-ul TCP pentru acceptarea clienților noi și menține referințe către toate controller-ele active. Implementează send și receive, utilități pentru controllere și gestionarea comunicării între server și clienți.

\textbf{ClientManagement} — Modul centralizat responsabil cu gestionarea thread-safe a clienților conectați. Folosește std::mutex pentru protejarea accesului concurrent la harta de clienți și implementează generarea automată de nume unice pentru fiecare client conectat (ex: arch\_0, ubuntu\_1).

\textbf{Controller-e Specializate}:
\begin{itemize}
\item \textit{ServerCommandController} — Interfață CLI administrativă care rulează într-un thread dedicat pentru citirea comenzilor din stdin
\item \textit{ServerFileController} — Gestionează transferul bidirecțional de fișiere cu encoding Base64
\item \textit{ServerPingController} — Implementează mecanismul de keep-alive prin UDP pentru detectarea clienților deconectați
\item \textit{ServerLogController} — Centralizează logging-ul thread-safe în /tmp/rat\_server.log și gestionează vizualizarea log-urilor prin spawn-area terminalelor
\end{itemize}

\subsection{Componente Client}

\textbf{ClientManager} — Orchestratorul principal care gestionează conexiunea la server și routează mesajele primite către controller-ele corespunzătoare.

\textbf{Controller-e Client}:
\begin{itemize}
\item \textit{ClientBashController} — Execută comenzi shell folosind popen() și returnează stdout/stderr
\item \textit{ClientFileController} — Implementează operațiile de upload/download cu Base64 encoding
\item \textit{ClientScreenshotController} — Capturează screenshot-uri folosind utilitare sistem (scrot, import, gnome-screenshot)
\item \textit{ClientPingController} — Răspunde la request-urile de keep-alive UDP
\item \textit{ClientKillController} — Gestionează deconectarea gracefully
\end{itemize}

\subsection{Utilități Comune}

Modulul Utils conține wrapper-e peste API-ul SFML Network:
\begin{itemize}
\item \texttt{TCPSocket} — Wrapper pentru socket-uri TCP cu API simplificat
\item \texttt{UDPSocket} — Wrapper pentru socket-uri UDP  
\item \texttt{Base64} — Librărie header-only pentru encoding/decoding date binare
\end{itemize}

\section{Concepte PSO Implementate}

\subsection{Gestionarea Memoriei}

Proiectul demonstrează management modern al memoriei în C++17, eliminând complet necesitatea malloc/free și new/delete manual:

\textbf{Smart Pointers}:
\begin{itemize}
\item \texttt{std::unique\_ptr} — Pentru ownership exclusiv (socket-uri, controller-e). La distrugerea obiectului, resursele sunt automat eliberate, prevenind memory leaks.
\item \texttt{std::shared\_ptr} — Pentru ServerManager (singleton) și resurse partajate între thread-uri cu reference counting automat.
\end{itemize}

\textbf{RAII Pattern} — Toate clasele implementează constructori/destructori care alocă/eliberează resurse automat. Exemplu în ServerManager:

\begin{verbatim}
ServerManager::~ServerManager() {
    stop();  // oprește thread-uri
    std::lock_guard<std::mutex> lock(mtx);
    cleanupResources();  // închide socket-uri
}
\end{verbatim}

La ieșirea din scope, destructorul este apelat automat, chiar și în cazul excepțiilor, asigurând cleanup complet al resurselor (socket-uri închise, thread-uri oprite cu join()).



\subsection{Pipe-uri pentru Execuție Comenzi}

ClientBashController demonstrează utilizarea pipe-urilor pentru execuția comenzilor shell și capturarea output-ului:

\begin{verbatim}
FILE* pipe = popen(cmd.c_str(), "r");
if (!pipe) {
    return json{{"error", "popen() failed"}};
}

std::string output;
char buffer[256];
while (fgets(buffer, sizeof(buffer), 
       pipe) != nullptr) {
    output += buffer;
}

int status = pclose(pipe);
int exitCode = WEXITSTATUS(status);

return json{
    {"output", output}, 
    {"exitcode", exitCode}
};
\end{verbatim}

Funcția popen() creează un proces copil prin fork() intern, execută comanda într-un shell și returnează un file descriptor conectat la stdout-ul procesului. Citirea se face printr-un pipe implicit creat de sistem, demonstrând comunicarea inter-proces unidirecțională.

\subsection{Gestionarea Proceselor}

ServerLogController::spawnTerminal() demonstrează utilizarea fork() și exec() pentru lansarea terminalelor externe pentru vizualizarea log-urilor:

\begin{verbatim}
pid_t pid = fork();
if (pid == 0) {  // proces copil
    setsid();     // detașare de terminal
    
    int devnull = open("/dev/null", O_WRONLY);
    dup2(devnull, STDOUT_FILENO);
    dup2(devnull, STDERR_FILENO);
    close(devnull);
    
    execlp("qterminal", "qterminal", "-e",
           "bash", "-c", cmd.c_str(), NULL);
    _exit(127);   // fallback dacă exec eșuează
}
// proces părinte continuă imediat
return pid;
\end{verbatim}

\textbf{Pași cheie}:
\begin{enumerate}
\item fork() creează proces copil identic
\item setsid() detașează copilul de terminalul părinte (daemon-izare)
\item dup2() redirecționează stdout/stderr către /dev/null
\item execlp() înlocuiește imaginea procesului cu terminalul dorit
\item Procesul părinte returnează imediat, permițând execuție asincronă
\end{enumerate}

\subsection{Thread-uri și Programare Concurentă}

Arhitectura multi-threading este esențială pentru funcționarea asincronă:

\begin{itemize}
\item \textbf{Thread principal} — Rulează loop-ul de acceptare clienți, blocând în accept() până la conectarea unui client nou
\item \textbf{Thread CLI} — ServerCommandController rulează un thread dedicat care citește comenzi din stdin prin std::getline(), permițând interacțiune administrativă simultană
\item \textbf{Thread-uri keep-alive} — ServerPingController poate avea thread-uri de ping periodic pentru monitorizarea conexiunii clienților
\end{itemize}

Crearea thread-urilor:
\begin{verbatim}
stdinThread = std::make_unique<std::thread>(
    &ServerCommandController::stdinLoop, this
);
\end{verbatim}

La distrugere, thread-urile sunt oprite prin flag-uri atomice (running = false) și sincronizate cu join() pentru a evita resource leaks:

\begin{verbatim}
void ServerCommandController::stop() {
    running = false;
    if (stdinThread && stdinThread->joinable()) {
        stdinThread->join();
    }
}
\end{verbatim}

\subsection{Sincronizare prin Mutex}

Accesul concurrent la resurse partajate este protejat prin std::mutex, demonstrând necesitatea sincronizării în sisteme multi-threaded:

\textbf{ClientManagement} folosește un mutex pentru protejarea hărții \texttt{clients} care stochează socket-urile active:

\begin{verbatim}
bool ClientManagement::addClient(
    const std::string& name, 
    std::unique_ptr<TCPSocket> socket) {
    std::lock_guard<std::mutex> lock(mtx);
    clients[name] = std::move(socket);
    return true;
}
\end{verbatim}

\textbf{ServerLogController} protejează queue-ul de log-uri \texttt{logs} cu un mutex dedicat \texttt{logMtx}:

\begin{verbatim}
void ServerLogController::pushLog(
    const std::string& msg) {
    std::lock_guard<std::mutex> lock(logMtx);
    logs.push(msg);
    
    std::ofstream ofs(getLogPath(), 
                      std::ios::app);
    if (ofs) ofs << msg << std::endl;
}
\end{verbatim}

Pattern-ul folosit este std::lock\_guard pentru RAII — mutexul este automat eliberat la ieșirea din scope, chiar și în cazul excepțiilor, prevenind deadlock-urile.

\subsection{Socket-uri TCP și UDP}

Comunicarea în rețea este realizată prin SFML Network, folosind două tipuri de socket-uri:

\textbf{Socket-uri TCP} — Folosite pentru comunicarea principală client-server. Serverul deschide un listener pe portul 5555 configurat în mod non-blocking pentru a accepta conexiuni noi fără a bloca execuția. Socket-urile client sunt configurate în mod blocking cu timeout de 5 secunde pentru a preveni deadlock-urile:

\begin{verbatim}
// Server - bind și listen
listener->bind(5555);
listener->setBlocking(false);  // non-blocking

// Loop accept
auto client = listener->accept();
if (client) {
    client->setBlocking(true);  
    // blocking cu timeout
    handleClient(std::move(client));
}
\end{verbatim}

\textbf{Socket-uri UDP} — Implementează mecanismul de keep-alive (ping/pong) pentru detectarea clienților deconectați. Alegearea UDP este justificată de natura stateless și overhead-ul redus pentru pachete mici periodice.

Implementarea wrapper-elor TCPSocket și UDPSocket demonstrează încapsularea API-ului de nivel jos SFML și oferă interfețe simplificate pentru operațiile de bind(), listen(), accept(), connect(), send() și receive().

\section{Protocol de Comunicare}

\subsection{Formatul Mesajelor}

Comunicarea client-server folosește JSON (librăria nlohmann::json) pentru structurarea mesajelor. Alegerea JSON este motivată de:
\begin{itemize}
\item \textbf{Lizibilitate} — mesajele pot fi debug-ate ușor
\item \textbf{Flexibilitate} — schema poate fi extinsă fără breaking changes
\item \textbf{Suport nativ C++} — librăria nlohmann oferă interfață idiomatică
\end{itemize}

Structura generală a unui mesaj:
\begin{verbatim}
{
  "controller": "bash|file|screenshot|...",
  "action": "execute|download|upload|...",
  ...parametri specifici...
}
\end{verbatim}

\subsection{Pattern Request-Response Sincron}

Serverul implementează comunicare sincronă prin două metode:

\texttt{sendRequest(clientName, json)} — Trimite cererea către client și returnează imediat status (success/failure pentru trimitere)

\texttt{receiveResponse(clientName, json\&, timeout)} — Blochează până primește răspuns de la client sau expiră timeout-ul (5-30 secunde în funcție de operație)

Acest pattern simplifică logica și elimină necesitatea callback-urilor asincrone, potrivit pentru scopul educațional al proiectului.

\section{Funcționalități Demonstrate}

\subsection{Execuție Comenzi Remote}

Comanda \texttt{bash} demonstrează utilizarea pipe-urilor și fork/exec:

\begin{verbatim}
Admin: bash arch_0 ls -la /tmp

Server -> Client:
  {"controller": "bash", "cmd": "ls -la /tmp"}

Client execută prin popen(), 
creează pipe intern

Client -> Server:
  {
    "output": "total 48\ndrwxr-xr-x...",
    "exitcode": 0
  }
\end{verbatim}

\subsection{Transfer Fișiere}

Operațiile \texttt{download} și \texttt{upload} folosesc Base64 encoding pentru transfer binar în JSON:

\textbf{Download}:
\begin{verbatim}
Admin: download arch_0 /etc/hostname ./h.txt

Server -> Client:
  {
    "controller": "file",
    "action": "download",
    "path": "/etc/hostname"
  }

Client: citește fișier + Base64::encode()

Client -> Server:
  {
    "success": true,
    "data": "YXJjaC1kZXNrdG9wCg==",
    "size": 13
  }

Server: Base64::decode() + write to ./h.txt
\end{verbatim}

Limitare: Fișiere max 50MB pentru a evita overflow-ul memoriei.

\subsection{Screenshot Remote}

Screenshot-ul demonstrează integrarea cu utilitare sistem și gestionarea proceselor externe:

\begin{verbatim}
Client detectează utilitar disponibil:
  1. Verifică which scrot/import/gnome-screenshot
  2. system("scrot /tmp/ss_<timestamp>.png")
  3. Citește PNG ca binar
  4. Base64::encode()

Client -> Server:
  {
    "success": true,
    "data": "iVBORw0KGgoAAAANS...",
    "size": 145230
  }

Server:
  1. Base64::decode()
  2. Write to /tmp/ss_arch_0_<time>.png
  3. fork() + exec() pentru xdg-open
\end{verbatim}

\subsection{Vizualizare Log-uri}

Comanda showlogs demonstrează fork/exec pentru spawn-area terminalelor:

\begin{verbatim}
Admin: showlogs

ServerLogController:
  1. Detectează terminal disponibil
     (qterminal, xfce4-terminal, konsole, ...)
  2. fork() proces copil
  3. setsid() pentru detașare
  4. dup2() pentru redirecționare stdout/stderr
  5. execlp() cu "bash -c tail -F log"

Terminal se deschide independent,
părinte continuă imediat
\end{verbatim}

\section{Testare și Rezultate}

\subsection{Mediu de Testare}

Sistemul a fost testat pe:
\begin{itemize}
\item \textbf{OS}: Kali Linux 
\item \textbf{Compilator}: GCC 13.2.1, C++17
\item \textbf{Librării}: SFML 2.6.0, nlohmann-json 3.11.2
\end{itemize}

\subsection{Scenarii de Testare}

\textbf{Test 1: Conectare multiplă și thread-safety}
\begin{itemize}
\item Pornit 5 clienți simultan
\item Verificat thread-safety în ClientManagement cu mutex
\item \textbf{Rezultat}: Toți clienții înregistrați corect, fără race conditions
\end{itemize}

\textbf{Test 2: Execuție comenzi concurente}
\begin{itemize}
\item Trimis comenzi bash către 3 clienți
\item Verificat că output-urile nu se amestecă
\item \textbf{Rezultat}: Fiecare răspuns asociat corect cu clientul său, sincronizare corectă
\end{itemize}

\textbf{Test 3: Transfer fișiere mari}
\begin{itemize}
\item Upload fișier 45MB (sub limita de 50MB)
\item Verificat integritate prin checksum MD5
\item \textbf{Rezultat}: Transfer reușit, date integre, Base64 encoding/decoding corect
\end{itemize}

\textbf{Test 4: Deconectare bruscă și cleanup}
\begin{itemize}
\item Închis forțat client (kill -9)
\item Verificat cleanup resurse pe server prin valgrind
\item \textbf{Rezultat}: Socket închis corect, RAII funcționează perfect, zero memory leaks
\end{itemize}


\subsection{Metrici de Performanță}

\begin{itemize}
\item \textbf{Latență medie}: 25-50ms pentru comenzi bash simple
\item \textbf{Throughput transfer}: ~8-10 MB/s pentru fișiere (limitat de overhead Base64 ~33\%)
\item \textbf{Memorie server}: ~15MB RAM cu 5 clienți conectați
\item \textbf{CPU idle}: <1\% când nu procesează cereri active
\end{itemize}

\section{Limitări și Îmbunătățiri}

\subsection{Limitări Actuale}

\textbf{Securitate}: Comunicarea este în plain-text fără autentificare sau criptare. Potrivit doar pentru medii educaționale controlate în rețele locale de încredere.

\textbf{Scalabilitate}: Arhitectura sincronă nu suportă eficient sute de clienți simultan. Fiecare request blochează până la răspuns.

\subsection{Îmbunătățiri Propuse}

\begin{itemize}
\item \textbf{TLS/SSL}: Adăugare criptare prin OpenSSL
\item \textbf{Asincron I/O}: Refactoring către epoll/io\_uring pentru scalabilitate
\item \textbf{Thread Pool}: Înlocuirea thread-urilor dedicate cu pool de workers
\end{itemize}

\subsection{Module Propuse}
\begin{itemize}
\item \textbf{KeyLogging}: Adăugare logging per client pentru tastele apasate
\item \textbf{NetworkCapture}: Salvare PCAP per client cu tshark
\item \textbf{LiveStream}: LiveStreamming service
\end{itemize}

\section{Concluzii}

Proiectul PSO-RAT demonstrează aplicarea practică a tuturor conceptelor fundamentale de sisteme de operare studiate în cadrul cursului la Academia Tehnică Militară:

\begin{itemize}
\item \textbf{Gestionarea memoriei} — Smart pointers și RAII elimină memory leaks
\item \textbf{Pipe-uri} — popen() pentru execuție comenzi și capturare output
\item \textbf{Procese} — fork(), exec(), setsid() pentru crearea proceselor independente
\item \textbf{Thread-uri} — Arhitectură multi-threaded cu execuție concurentă
\item \textbf{Mutex} — Sincronizare thread-safe cu std::lock\_guard
\item \textbf{Socket-uri} — TCP pentru comunicare, UDP pentru keep-alive
\end{itemize}

Implementarea în C++17 respectă principiile moderne de programare sistem, demonstrând înțelegerea profundă a modului în care conceptele teoretice se traduc în implementări practice funcționale.

Deși sistemul are limitări de securitate și scalabilitate (inerente unui proiect educațional), arhitectura modulară permite extindere viitoare. Codul sursă este disponibil pe GitHub pentru studiu în medii controlate.

\section*{Mulțumiri}

Acest proiect a fost realizat în cadrul cursului Proiectarea Sistemelor de Operare la Academia Tehnică Militară „Ferdinand I", semestrul 2, anul universitar 2025-2026.

Mulțumiri cadrelor didactice pentru ghidarea conceptelor de sisteme de operare și comunității open-source pentru librăriile SFML și nlohmann::json care au facilitat implementarea.

\begin{thebibliography}{00}
\bibitem{tanenbaum2014}
A.~S.~Tanenbaum și H.~Bos, 
``Modern Operating Systems,'' 
ediția a 4-a, Pearson, 2014.

\bibitem{silberschatz2018}
A.~Silberschatz, P.~B.~Galvin, și G.~Gagne,
``Operating System Concepts,''
ediția a 10-a, Wiley, 2018.

\bibitem{stevens2013}
W.~R.~Stevens și S.~A.~Rago,
``Advanced Programming in the UNIX Environment,''
ediția a 3-a, Addison-Wesley, 2013.

\bibitem{kerrisk2010}
M.~Kerrisk,
``The Linux Programming Interface: A Linux and UNIX System Programming Handbook,''
No Starch Press, 2010.

\bibitem{sfmlnetwork}
SFML Development Team,
``SFML Network Module Documentation,''
[Online]. Disponibil: https://www.sfml-dev.org/

\bibitem{nlohmannjson}
N.~Lohmann,
``JSON for Modern C++,''
[Online]. Disponibil: https://json.nlohmann.me/

\bibitem{posixthreads}
IEEE și The Open Group,
``POSIX Threads Programming,''
[Online]. Disponibil: https://pubs.opengroup.org/

\bibitem{cppref}
C++ Reference,
``C++ Standard Library,''
[Online]. Disponibil: https://en.cppreference.com/

\end{thebibliography}

\end{document}
